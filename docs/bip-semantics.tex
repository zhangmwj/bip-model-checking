\documentclass[a4paper]{article}
\usepackage[utf8]{inputenc}
\usepackage[T1]{fontenc}
\usepackage{amsmath, amsthm, amssymb}
\usepackage{mathpartir}
\begin{document}
\section*{Atomic Components}
\subsection*{Formal syntax}
Let $V$ be a set of variables, $X$ a set of clocks and $\mathcal{E}(V)$ the set of expressions where variables are in
$V$. Variables can be integer or Boolean and expressions can use comparison operators ($<,\leq,>,\geq,=,\neq$),
arithmetics operators ($+,-,*,/$ (euclidean division)$,\%$), logical Boolean operators ($!, \&\&, ||$), Boolean bitwise
operators (\textasciitilde$, \&, |,$\textasciicircum) and function calls with returned value.\\
A \textbf{clock constraint} over $V \cup X$ is a conjunction of atomic constraints of the form
$x \sim n$ or $x - y \sim n$ with $x,y \in X, \sim \in \{\leq, <, \geq, >\}$ and $n \in \mathcal{E}(V)$ evaluates to a
natural number without side effects. \\
A \textbf{guard} over $V \cup X$ is a (possibly empty) conjunction of atomic guard of the form $c$ or $b$ or $c \vee b$
with $c$ a clock constraint and $b \in \mathcal{E}(V)$ evaluates to a Boolean value without side effects. We write
$\mathcal{G}(V, X)$ to denote the set of guards over $V \cup X$ and $\mathcal{G}(V) = \mathcal{G}(V, \emptyset)$. \\
An \textbf{invariant} over $V \cup X$ is defined as a guard where $\sim$ is restricted to $\{\leq, <\}$. We write
$\mathcal{I}(V, X)$ to denote the set of invariants over $V \cup X$. \\
An \textbf{update} on $V \cup X$ is a sequence of assignments of the form $x = n$ or $v = e$ or $f(p)$ with
$x \in X, n \in \mathcal{E}(V)$ evaluates to a natural number, $v \in V, e \in \mathcal{E}(V)$, $p$ a list of variables
in $V$ and $f$ a function that can assign values to variables in $p$. We write $\mathcal{F}(V,X)$ to denote the set of
updates on $V \cup X$ and $\mathcal{F}(V) = \mathcal{F}(V,\emptyset)$.\\
A \textbf{port} is a pair $\langle p, x_p \rangle$, denoted $p$, where:
\begin{itemize}
  \item $p$ is an identifier,
  \item $x_p$ is a set of variables.
\end{itemize}
An \textbf{atomic component} is a tuple $A = \langle V_A, X_A, P_A, L_A, l_A^0, E_A, I_A \rangle$ where:
\begin{itemize}
  \item $V_A$ is a finite set of variables with $v_0$ the initial value of $v$ for each $v \in V_A$,
  \item $X_A$ is a finite set of clocks,
  \item $P_A$ is a finite set of ports, each $p \in P_A$ is associated with a set of variables $x_p \subseteq V_A$,
  \item $L_A$ is a finite set of locations,
  \item $l_A^0 \in L_A$ is the initial location,
  \item $E_A$ is a finite set of transitions, each transition is a tuple $\langle l, p, g, f, l', u \rangle$ with
        $l,l' \in L_A, p \in P_A, f \in \mathcal{F}(V_A, X_A)$, $u \in \{e,l\}$ an urgency ($e$ for eager and $l$ for
        lazy) and $g \in \mathcal{G}(V_A, X_A)$ if $u = l$ and $g \in \mathcal{G}(V_A)$ otherwise,
  \item $I_A: L_A \to \mathcal{I}(V_A,X_A)$ is a function that assigns an invariant to each location.
\end{itemize}
We denote by $\mathcal{T}(A,p)=\{\langle l, p', g, f, l',u \rangle \in E_{A} \:\vert\: p'=p\}$ the set of transitions
of $p$ in $A$.
\subsection*{Semantics}
A \textbf{valuation} $v: V \cup X \to \mathcal{D}$ is a function mapping variables in $V$ and clocks in $X$ to their
codomain, we denote by $\mathcal{D}$ the union of these codomains.
For $\delta \in \mathbb{R}^+, v+\delta$ is defined as $(v+\delta) (x) =
  \begin{cases}
    v(x)          & \text{if $x \in V$} \\
    v(x) + \delta & \text{if $x \in X$}
  \end{cases}.$
For $f \in \mathcal{F}(V, X), f(v)$ is the valuation $v$ updated by $f$. \\
For two functions $v: X \to Y$ and $v': X' \to Y'$, the \textbf{substitution function} noted
$v/v': X \cup X' \to Y \cup Y'$ is defined as $(v/v') (x) =
  \begin{cases}
    v'(x) & \text{if } x \in X'             \\
    v(x)  & \text{if } x \in X \setminus X'
  \end{cases}$. \\
For a function $f: X \to Y$ and a set $X' \subseteq X$, the \textbf{restriction} of $f$ to $X'$ noted
$f \vert_{X'}: X' \to Y$ is defined as $f \vert_{X'}(x) = f(x)$ for $x \in X'$.\\
For two sets $X$ and $Y$, the set of all functions $f: X \to Y$ is denoted $Y^X$. \\
The semantics of an atomic component $A$ is a LTS $\langle Q_A, q_A^0, \Sigma_A, \xrightarrow{}_A \rangle$ where:
\begin{itemize}
  \item $Q_A$ is a set of states of the form $\langle l, v \rangle$ with $l \in L_A$ and
        $v \in \mathcal{D}^{V_A \cup X_A}$ a valuation of variables and clocks of $A$,
  \item $q_A^0$ is the initial state $\langle l^0, v_0 \rangle$ with $v_0(x) = x_0$ for each $x \in V_A$ and
        $v_0(x) = 0$ for each $x \in X_A$,
  \item $\Sigma_A$ is the set of labels of the form $d$ or $\langle p, u \rangle$ with
        $d \in \mathbb{R}_{>0}, p \in P_A$ and $u \in \{e,l\}$ an urgency
  \item $\xrightarrow{}_A$ is the transition relation defined by the rules:
        \begin{mathpar}
          \inferrule[port]{
            \langle l, p, g, f, l', u \rangle \in E_A \\
            v \vDash g \\
            v' = f(v)\\
            v' \vDash I_A(l')
          } {
            \langle l, v \rangle \xrightarrow{\langle p, u \rangle}_A \langle l', v' \rangle
          }
        \end{mathpar}
        \begin{mathpar}
          \inferrule[delay]{
            v' = v+\delta \\
            v' \vDash I_A(l) \\
            \neg \bigl(\exists q \in Q_A, \exists p \in P_A:
            \langle l, v \rangle \xrightarrow{\langle p, e \rangle}_A q\bigr)~(1)
          } {
            \langle l, v \rangle \xrightarrow{\delta}_A \langle l, v' \rangle
          }
        \end{mathpar}
        $(1)$ expresses that there is no port transition from $\langle l, v \rangle$ with an eager urgency
\end{itemize}
\section*{Composite Components}
\subsection*{Formal syntax}
Let $(A_1, \ldots, A_n)$ be a vector of atomic components and $I \subseteq [1,n]$. \\
A \textbf{connector} is a tuple $\gamma = \langle T_{\gamma}, S_{\gamma}, A_{\gamma} \rangle$ where:
\begin{itemize}
  \item $T_{\gamma}$ and $S_{\gamma}$ are two disjoint sets of ports with
        $T_{\gamma} \cup S_{\gamma} = \{p_i \mid i \in I\}$ and $p_i \in P_{A_i}$ for each $i \in I$, $I$ defines the
        indices of the atomics components from where the ports of the connector come and there is at most one port per
        atomic component in the connector,
  \item $A_{\gamma}$ is a set of actions, each action is a tuple $\langle P, g, f \rangle$ with
        \begin{itemize}
          \item $P = \{p_j \mid j \in J\} \subseteq T_{\gamma} \cup S_{\gamma}$ for a $J \subseteq I$ where either
                $P \cap T_{\gamma} \neq \emptyset$ or $P = T_{\gamma} \cup S_{\gamma}$, $T_{\gamma}$ is a set
                of trigger ports that can initiate an interaction without synchronizing and $S_{\gamma}$ is a set of
                synchron ports that needs synchronization with other ports to initiate an interaction,
          \item $\forall i \in J: \bigl(\exists \langle l, p, g, f, l', u \rangle \in \mathcal{T}(A_i, p_i): u = e
                  \bigr) \Rightarrow \forall j \in J \setminus \{i\}, \forall \langle l, p, g, f, l', u \rangle \in
                  \mathcal{T}(A_j, p_j): g \in \mathcal{G}(V_{A_j})$, i.e. if a port has an eager
                transition then the transitions of the other ports cannot have clocks in their guard,
          \item $g \in \mathcal{G}(\bigcup_{j \in J} x_{p_j})$,
          \item $f \in \mathcal{F}(\bigcup_{j \in J} x_{p_j})$.
        \end{itemize}
\end{itemize}
We denote by $\mathcal{A}(\gamma) = \{P \mid \langle P, g ,f \rangle \in A_{\gamma}\}$ the set of \textbf{interactions}
of $\gamma$. \\
A \textbf{priority} model $\pi$ is a partial order on a set of connector. \\
A \textbf{composite component} is tuple $C = \langle (A_1, \ldots, A_n), \Gamma_C, \pi_C \rangle, n \in \mathbb{N}$
where:
\begin{itemize}
  \item $(A_1, \ldots, A_n)$ is a vector of atomic components with the sets in $\{V_{A_i} \mid 1 \leq i \leq n\}$ and
        $\{X_{A_i} \mid 1 \leq i \leq n\}$ pairwise disjoint,
  \item $\Gamma_C$ is a set of connectors on $(A_1, \ldots, A_n)$,
  \item $\pi_C$ a priority model on $\Gamma_C$.
\end{itemize}
We denote by $\mathcal{B}(C) = \bigcup_{1 \leq i \leq n} P_{A_i} \setminus \bigcup_{\gamma \in \Gamma_C} T_{\gamma}
  \cup S_{\gamma}$ the set of \textbf{internal ports} of $C$.
\subsection*{Semantics}
The semantics of a composite component $C$ is a LTS $\langle Q_C, q^0_C, \Sigma_C, \xrightarrow{}_C \rangle$ where:
\begin{itemize}
  \item $Q_C = \prod_{1 \leq i \leq n} Q_{A_i}$,
  \item $q^0_C = \langle q^0_{A_1}, \ldots, q^0_{A_n} \rangle$,
  \item $\Sigma_C = \mathbb{R}_{>0} \cup \bigl(\bigcup_{\gamma \in \Gamma_C} \mathcal{A}(\gamma) \cup \mathcal{B}(C)
          \bigr) \times \{e,l\}$,
  \item $\xrightarrow{}_C$ is defined by the rules:
        \begin{mathpar}
          \inferrule[internal port]{
            i \in [1,n] \\
            p \in \mathcal{B}(C) \cap P_{A_i} \\
            q_i \xrightarrow{\langle p, u \rangle}_{A_i} q_i' \\
            \forall j \in [1,n] \setminus \{i\}: q_j = q_j'
          } {
            \langle q_1, \ldots, q_n \rangle \xrightarrow{\langle p, u \rangle}_C \langle q_1', \ldots, q_n' \rangle
          }
        \end{mathpar}
        \begin{mathpar}
          \inferrule[interaction]{
            I \subseteq [1, n] \\
            a = \{p_i \mid i \in I \} \wedge p_i \in P_{A_i} \\
            \exists \gamma \in \Gamma_C: \langle a, g, f \rangle \in A_{\gamma}\\
            v \vDash g \\
            \forall i \in I, \exists u_i \in \{e,l\}: \langle l_i, v_i'' \rangle
            \xrightarrow{\langle p_i, u_i \rangle}_{A_i} q_i' \wedge v_i'' = v_i / \bigl(f \vert_{x_{p_i}}(v)\bigr)~(2)\\
            u = \text{$\begin{cases}
                  e & \text{if } \exists i \in I : u_i = e \\
                  l & \text{otherwise}
                \end{cases}$} \\
            \forall i \not\in I : q_i = q_i' \\
            \neg \bigl(\exists \gamma' \in \Gamma_C, \exists a' \in \mathcal{A}(\gamma'), \exists u \in \{e,l\},
            \exists Q \in Q_C: \gamma \prec_{\pi_C} \gamma' \wedge \langle q_1, \ldots, q_n \rangle
            \xrightarrow{\langle a', u \rangle}_C Q\bigr)~(3) \\
            \neg \bigl(\exists a' \in \mathcal{A}(\gamma), \exists u \in \{e,l\}, \exists Q \in Q_C: a \subset a'
            \wedge \langle q_1, \ldots, q_n \rangle \xrightarrow{\langle a', u \rangle}_C Q\bigr)
            ~(4)
          } {
            \langle q_1, \ldots q_n \rangle \xrightarrow{\langle a, u \rangle}_C \langle q_1', \ldots, q_n' \rangle
          }
        \end{mathpar}
        where $v = v_I \vert_{\bigcup_{i \in I} x_{p_i}}$ with $v_I$ defined as $v_I(x) = v_i(x)$ for each
        $x \in V_{A_i}$ if $i \in I$, it is the valuation of variables of the
        ports in $a$ \\
        $(2)$ $v_i''$ is the valuation of variables in $A_i$ updated by $f$ \\
        $(3)$ and $(4)$ express that there is no interaction with a higher priority or including $a$ from
        $\langle q_1, \ldots, q_n \rangle$\\
        \begin{mathpar}
          \inferrule[delay]{
            \forall i \in [1, n]: q_i \xrightarrow{\delta}_{A_i} q_i' \\
            \neg \bigl(\exists p \in \bigcup_{\gamma \in \Gamma_C} \mathcal{A}(\gamma) \cup \mathcal{B}(C),
            \exists Q \in Q_C: \langle q_1, \ldots, q_n \rangle \xrightarrow{\langle p, e \rangle}_C Q\bigr)~(5)
          } {
            \langle q_1, \ldots q_n \rangle \xrightarrow{\delta}_C \langle q_1', \ldots, q_n' \rangle
          }
        \end{mathpar}
        $(5)$ expresses that there is no internal port or interaction transition from
        $\langle q_1, \ldots q_n \rangle$ with an eager urgency
\end{itemize}
\section*{Differences}
\begin{itemize}
  \item No hierarchy:
        \begin{itemize}
          \item connectors have no exported port
          \item components have no compound component, exported data and exported port
        \end{itemize}
  \item No guard on priorities and only priorities on connectors:
        \begin{itemize}
          \item atoms have no priority on ports, export data and merged ports
          \item priority rules in components are in the form \texttt{I < J} where \texttt{I} and \texttt{J} are in the
                form \texttt{C:*} or \texttt{*:*} where \texttt{C} is a connector
        \end{itemize}
  \item There is no sub-expression in the expressions of the interactions in connectors
  \item No delayable urgency, invariant from and resume
  \item Guard and invariant conditions are restricted to the grammar: \\
        $\begin{aligned}
            G & ::= c \:\vert\: b \:\vert\: c \vee b \:\vert\: G \wedge G \\
            c & ::= x \sim n \:\vert\: x - y \sim n \:\vert\: c\wedge c
          \end{aligned}$ \\
        where $b$ is an expression that evaluates to a Boolean value, $x,y$ are clocks, $n$ is an expression that
        evaluates to a natural number, $\sim \in \{<, \leq, >, \geq \}$ for guards and $\sim \in \{<, \leq \}$ for
        invariants
  \item No internal transition
  \item No if-then-else statement in actions
  \item Only Boolean and integer variables
\end{itemize}
\end{document}
